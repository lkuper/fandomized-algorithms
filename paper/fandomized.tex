\documentclass[9pt]{sigplanconf}
\usepackage[utf8]{inputenc}
\usepackage{url}
\usepackage{latexsym}
\usepackage{textcomp}
\usepackage{amsfonts}
\usepackage{amsmath, amsthm, amssymb}
\usepackage{float}
\usepackage{graphicx}
\usepackage{listings}
\usepackage{etoolbox}
\usepackage{float}
\floatstyle{boxed}
\restylefloat{figure}

%% Patch the copyright box to make it go away.
%% Better: challenge the notion of copyright at all, put something about how
%% ACH believes that copyright is theft.
%% An interesting anecdote: the style file that has the ACM copyright notice is
%% CC-BY.
\makeatletter
\patchcmd{\maketitle}{\@copyrightspace}{}{}{}
\makeatother

\begin{document}

\title{Fandomized Algorithms and Fandom Number Generation}

\authorinfo{Lindsey Kuper \and Alex Rudnick}
           {School of Transformative Works, Indiana University}
           {\{lkuper, alexr\}@cs.indiana.edu}

\maketitle

\begin{abstract}
We introduce the concept of \emph{fandomness} and of fandomized algorithms, and
demonstrate their application in a practical fandom number generator.
\end{abstract}

\begin{figure}[bl]
\begin{verbatim}
$ ./fandom_number_generator.py 
No numbers in fanwork #346401
YOUR FANDOM NUMBER: 286
from fanwork #369546
http://archiveofourown.org/works/369546
\end{verbatim}
\caption{Sampling a fandom number from AOOO}
\end{figure}

\section{Introduction}

fandom variable
fandom number
fentropy
fandomized algorithm
MLP-hard
fandomly
fentropic
fenergy (?)
Dreamwidth
OTPs

\section{Fandomness and fandom variables}
A \emph{fandom variable} can take on a \emph{fandom number}, but the generation
of fandom numbers requires a source of \emph{fentropy}. Thankfully, there exist
fentropic processes in nature \emph{Fentropy} is a measure of the fannishness
of a fandom variable over time.  Computing the fentropy of a fandom variable
exactly is in general MLP-Hard, although approximations are possible.  The
world's technological capacity to store and communicate fentropic information
has increased since the advent of the information age, especially since
Dreamwidth launched.

\section{A practical fandom number generator}
We have developed a practical algorithm and implementation\footnote{
\url{http://github.com/lkuper/fandomized-algorithms}} for
generating fandom numbers, a key component for any fandomized
algorithm. A naturally occurring source of fentropy, Archive of Our
Own\footnote{Google for ``archive of our own"; do you not know how to
  do web searches?\footnotemark}
\footnotetext{Oh, fine.  It's \cite{ao3}.}, supplies an
ever-increasing amount of fandomness, certainly more than the current
global demand for fentropy to power fandomized algorithms. As
fandomized algorithms become more broadly deployed, further sources of
fandomness may be required -- we may have to mine THE DARK
SOCIAL\footnote{THE DARK SOCIAL} -- but we will burn that bridge when
we get to it.

Our practical fandom number generator downloads a pseudo-fandomly-selected
transformative work from AOOO, locates all of the base-10 numbers in it, and
then returns one of them at fandom. If for some reason there are no fandom
numbers present in a given transformative work, we simply try another
transformative work until we find one.

This work would not be canon without the public availability of sources of
fentropy; the open publishing and reuse rights of the transformative works on
AOOO enable us to produce our own transformative works.

\section{OTPs}
OTPs are one of the most important applications of fandom numbers.  An
OTP is when you put a character together with another character from a
secret fandom pad, the one with whom it truly belongs (mod 26).  The
true love of an OTP has been proven to be impossible to break.

\section{Markov fandom fields}
We may also wish to do inference over communities of interacting fandom
variables using a Markov fandom field and the \emph{headcanon propagation}
algorithm, although it is MLP-hard in most cases.  However, we can perform
approximate inference with loopy headcanon propagation.  Fandom-wanking is not
guaranteed to terminate in this case, and a consistent community-wide headcanon
may not emerge.\footnote{The alert reader may have noticed that Tumblr is a
platform for human computation \cite{luisvonahn}, performing loopy headcanon
propagation at scale. Most Tumblr traffic is used for applications in protein
folding and computational geophysics.}

\section{Outro}
As the dictum about software development goes, ``shipping code wins". We have
accepted this headcanon, but we realize that many ponies in the computer
science community remain squicked by it.

As such, we have provided an overview of \emph{fandomized algorithms},
\emph{fandomness}, and \emph{fandom variables}, and explored some of their
applications in computing. In upcoming seasons, we expect that it will be
revealed that fandom/CS is the OTP.

\bibliographystyle{abbrvnat}
\bibliography{fandomized}

\end{document}
