\documentclass[9pt]{sigplanconf}
\usepackage[utf8]{inputenc}
\usepackage{url}
\usepackage{latexsym}
\usepackage{textcomp}
\usepackage{amsfonts}
\usepackage{amsmath, amsthm, amssymb}
\usepackage{float}
\usepackage{graphicx}
\usepackage{listings}
\usepackage{etoolbox}
\usepackage{float}
\floatstyle{boxed}
\restylefloat{figure}

%% Patch the copyright box to make it go away.
%% Better: challenge the notion of copyright at all, put something about how
%% ACH believes that copyright is theft.
%% An interesting anecdote: the style file that has the ACM copyright notice is
%% CC-BY.
\makeatletter
\patchcmd{\maketitle}{\@copyrightspace}{}{}{}
\makeatother

\begin{document}

\title{Fandomized Algorithms and Fandom Number Generation}

\authorinfo{Lindsey Kuper \and Alex Rudnick}
           {School of Transformative Works, Indiana University}
           {\{lkuper, alexr\}@cs.indiana.edu}

\maketitle

\begin{abstract}
We introduce the concepts of \emph{fandomness} and fandomized algorithms,
discuss some of their applications, and demonstrate a practical fandom number
generator.
\end{abstract}

\category{Pairing/Characters}{fandom/CS}{}
\category{Rating}{PG-13}{}

\begin{figure}[bl]
\begin{verbatim}
$ ./fandom_number_generator.py 
No numbers in fanwork #346401
YOUR FANDOM NUMBER: 286
from fanwork #369546
http://archiveofourown.org/works/369546
\end{verbatim}
\caption{Sampling a fandom number from AO3.}
\end{figure}

\section{Spoiler warning}
Fandomized algorithms make use of fandom numbers and fentropy to
perform useful, or at least emotionally satisfying, computation. Here
we discuss some of the most prominent applications for fandomized
algorithms.

\section{Fandomness and fandom variables}
A \emph{fandom variable} can take on a \emph{fandom number}, but the
generation of fandom numbers requires a source of
\emph{fentropy}. Thankfully, there exist fentropic processes in
nature, and we can typically sample from them over the
Internet. \emph{Fentropy} is a measure of the fannishness of a fandom
variable over time.  The world's technological capacity to store and
communicate fentropic information has increased since the advent of
the information age, especially since Dreamwidth launched.

\section{OTPs}
OTPs are one of the most important applications of fandom numbers.  In
an OTP, a character is combined with another character from a secret
fandom pad, the one with whom it truly belongs (mod 26).  For
characters $c_1$ and $c_2$, we denote such a pairing as $c_1/c_2$.  If
the OTP key material is truly fandom (sampling from {\tt
  /dev/ufandom}, for instance, may be insufficiently fandom), the true
love of an OTP has been proven impossible to break.

\section{Markov fandom fields}
We may also wish to do inference over communities of interacting
fandom variables using a Markov fandom field and the \emph{headcanon
  propagation} algorithm, although it is MLP-hard in most cases.
However, we can perform approximate inference with loopy headcanon
propagation.  Fandom-wanking is not guaranteed to terminate in this
case, and a consistent community-wide headcanon may not
emerge.\footnote{The alert reader may have noticed that Tumblr is a
  platform for human computation \cite{luisvonahn}, performing loopy
  headcanon propagation at scale. Most Tumblr traffic is used for
  applications in protein folding and computational geophysics.}

\section{A practical fandom number generator}
We have developed a practical algorithm and implementation for
generating fandom numbers, which are a key component for any
fandomized algorithm. Our fandom number generator is available at:
\begin{center}
\url{http://github.com/lkuper/fandomized-algorithms}
\end{center}
\noindent A naturally occurring source of fentropy, \textit{Archive of
  Our Own} (AO3), supplies an ever-increasing amount of fandomness,
certainly more than the current global demand for fentropy to power
fandomized algorithms.\footnote{Google for ``archive of our own"; do
  you not know how to do web searches?\footnotemark}\footnotetext{Oh,
  fine.  \cite{ao3}.} As fandomized algorithms become more broadly
deployed, further sources of fandomness may be required.

Our practical fandom number generator downloads a
pseudo-fandomly-selected transformative work from AO3, locates all of
the base-10 numbers in it, and then returns one of them at fandom. If
for some reason there are no fandom numbers present in a given
transformative work, we simply try another transformative work until
we find one.

This work would not be canon without the public availability of
sources of fentropy; the open publishing and reuse rights of the
transformative works on AO3 enable us to transform these
transformative works into transformative works of our own.

\section{Season finale}
As the dictum about software development goes, ``shipping code
wins". We have accepted this headcanon, but we realize that many
ponies in the computer science community remain squicked by it.

As such, we have provided an overview of \emph{fandomized algorithms},
\emph{fandomness}, and \emph{fandom variables}, and explored some of
their applications in computing. In upcoming seasons, we expect that
it will be revealed that fandom/CS is the OTP.

\section*{Acknowledgments}
We would like to thank our beta readers.

\bibliographystyle{abbrvnat}
\bibliography{fandomized}

\end{document}
